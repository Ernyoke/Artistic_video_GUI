\documentclass[12pt, a4paper, oneside]{book}

%---------------------------------------------------------   
\usepackage[english,romanian,magyar]{babel}       
\usepackage[utf8]{inputenc}
\usepackage[a4,center,axes]{crop}
\usepackage{calc}
\usepackage{t1enc}
\usepackage{amsthm}
\usepackage{rotating}
\usepackage{amssymb}
\usepackage{lscape}
\usepackage{anysize}
\usepackage{setspace} 
\usepackage{comment}
\usepackage{graphicx}
\usepackage{setspace} 
\usepackage{tocloft}
\usepackage{indentfirst}
\usepackage{url}
\usepackage[top=3cm, bottom=2cm, left=3cm, right=2cm]{geometry}

%---------------------------------------------------------
%%
%%
%% alapformázások
%%
%%
\sloppy                        % sorkizárás kezelése
\clubpenalty = 10000           % árvasorok
\widowpenalty = 10000          % fatyúsorok
\raggedbottom                  % függõleges kizárás az oldalon
\setcounter{secnumdepth}{3}    % alcimek számozási mélysége
\setcounter{tocdepth}{3}       % tartalomjegyzék mélysége
\brokenpenalty = 10000         % lap aljai elválasztások tiltása
\doublehyphendemerits = 80000  % egymást követõ elválasztások

%\marginsize{2cm}{2cm}{2cm}{2cm} %marók beállítása
\onehalfspacing %sorközök megaddása

\theoremstyle{tetel}
\newtheorem{mydef}{értelmezés}[chapter]
\newtheorem{megj}{megjegyzés}[chapter]

\def\partro#1{\foreignlanguage{romanian}{\addcontentsline{tro}{part}{%\if@mainmatter\protect\numberline{\thechapter.}\fi
\MakeUppercase{#1}}}}
\def\chapterro#1{\foreignlanguage{romanian}{\addcontentsline{tro}{chapter}{\if@mainmatter\protect\numberline{\thechapter.}\fi#1}}}
\def\sectionro#1{\foreignlanguage{romanian}{\addcontentsline{tro}{section}{\protect\numberline{\thesection.}#1}}}
\def\subsectionro#1{\foreignlanguage{romanian}{\addcontentsline{tro}{subsection}{\protect\numberline{\thesubsection.}#1}}}
\def\subsubsectionro#1{\foreignlanguage{romanian}{\addcontentsline{tro}{subsubsection}{\protect\numberline{\thesubsubsection.}#1}}}

%roman tartalom def
%---------------------------------------------------------
\newcommand{\nombreindice}{Cuprins}
\newlistof{indice}{tce}{\nombreindice}

\newcommand\capterro[1]{%
  \addcontentsline{tce}{chapter}{\protect\makebox[1.3em][l]{\thechapter.}#1}}
\newcommand\secro[1]{%
  \addcontentsline{tce}{section}{\protect\makebox[2.8em][l]{\thesection.}#1}}
\newcommand\ssecro[1]{%
\addcontentsline{tce}{subsection}{\protect\makebox[3em][l]{\thesubsection.}#1}}

%angol tartalom def
%---------------------------------------------------------

\newcommand{\tcontents}{Table Of Contents}
\newlistof{indiceen}{tcen}{\tcontents}

\newcommand\capteren[1]{%
  \addcontentsline{tcen}{chapter}{\protect\makebox[1.3em][l]{\thechapter.}#1}}
\newcommand\secen[1]{%
  \addcontentsline{tcen}{section}{\protect\makebox[2.8em][l]{\thesection.}#1}}
\newcommand\ssecen[1]{%
\addcontentsline{tcen}{subsection}{\protect\makebox[3em][l]{\thesubsection.}#1}}

%---------------------------------------------------------
\newcommand*{\field}[1]{\mathbb{#1}}

%---------------------------------------------------------

%---------------------------------------------------------
\begin{document}
\pagenumbering{arabic}

%magyar borito
%--------------------------------------------------------
\newpage
\thispagestyle{empty}
\begin{center}
    \Large SAPIENTIA ERDÉLYI MAGYAR TUDOMÁNYEGYETEM\\
    \Large MŰSZAKI ÉS HUMÁNTUDOMÁNYOK KAR, MAROSVÁSÁRHELY\\
    \Large SZOFTVERFEJLESZTÉS SZAK\\
\end{center}

\begin{center}
 	\vspace{2cm}\LARGE \textbf{DeepArt}\\
	 \vspace{1cm}\LARGE \textbf{MESTERI DISSZERTÁCIÓ}\\
\end{center}

\vspace{2cm}
\begin{figure}[htb]
\hspace{5.7cm}\includegraphics[bb = 0 0 160 160]{sapi.jpg}
\end{figure}

\vspace{2cm}
\begin{center}
\begin{tabular}{lcccccccccccl}
    TÉMAVEZETŐ:&&&&&&& &&&&&SZERZŐ:\\
     dr. Iclănzan Dávid&&&&&& &&&&&&Szilágyi Ervin\\
	Egyetemi tanár
\end{tabular}
\end{center}

\begin{center}
    \vspace{0.5cm}\textbf{2017 Június}
\end{center}
\vspace*{\fill}
%roman borito
%-------------------------------------------------------------
\newpage
\thispagestyle{empty}
\begin{center}
    %\Large UNIVERSITATEA BABE\c{S}-BOLYAI CLUJ-NAPOCA\\
    \Large UNIVERSITATEA SAPIENTIA TÂRGU-MURE\c{S}\\
    \Large FACULTATEA DE \c{S}TIIN\c{T}E TEHNICE \c{S}I UMANISTE\\
    \Large SPECIALIZAREA DEZVOLTARE DE SOFTWARE\\
\end{center}

\begin{center}
    \vspace{3cm}\LARGE \textbf{DeepArt}\\
    \vspace{1cm}\LARGE\textbf{Lucrare de master}\\
\end{center}

\vspace{2cm}
\begin{figure}[htb]
\hspace{5.7cm}\includegraphics[bb = 0 0 160 160]{sapi.jpg}
\end{figure}

\vspace{2cm}
\begin{center}
\begin{tabular}{lcccccccccccl}
    Coordonator \c{s}tiin\c{t}ific:&&&&&&& &&&&&Absolvent:\\
     dr. Iclănzan Dávid&&&&&& &&&&&&Szilágyi Ervin\\

\end{tabular}
\end{center}

\begin{center}
    \vspace{1cm}\textbf{2017 Iunie}
\end{center}

%angol borito
%-------------------------------------------------------------

\newpage
\thispagestyle{empty}
\begin{center}
    %\Large BABE\c{S}-BOLYAI UNIVERSITY CLUJ-NAPOCA\\
    \Large SAPIENTIA UNIVERSITY TÂRGU MURE\c{S}\\
    \Large FACULTY OF TECHNICAL AND HUMAN SCIENCES\\
    \Large SOFTWARE DEVELOPMENT SPECIALIZATION\\
\end{center}

\begin{center}
    \vspace{3cm}\LARGE \textbf{DeepArt}\\
    \vspace{1cm}\LARGE \textbf{Master Thesis}\\
\end{center}

\vspace{2cm}
\begin{figure}[htb]
\hspace{5.7cm}\includegraphics[bb = 0 0 160 160]{sapi.jpg}
\end{figure}

\vspace{2cm}
\begin{center}
\begin{tabular}{lcccccccccccl}
     Advisor: & & &&&& &&&&&& Student:\\
     dr. Iclănzan Dávid &&&&&& &&&&&& Szilágyi Ervin\\
\end{tabular}
\end{center}

\begin{center}
    \vspace{1cm}\textbf{2017 June}
\end{center}

%eredetisegi nyilatkozat
%-------------------------------------------------------------
\newpage
\thispagestyle{empty}
eredetisegi nyilatkozat
%kivonat magyar
%-------------------------------------------------------------
\newpage
\thispagestyle{empty}
\begin{center}
    \Large KIVONAT
\end{center}

kivonat

\begin{flushright}
\textbf{Szilagyi Ervin,}\\
.........................
\end{flushright}

%kivonat roman
%-------------------------------------------------------------
\newpage
\thispagestyle{empty}
\begin{center}
    \Large ABSTRACT
\end{center}

abstract

\begin{flushright}
\textbf{Szilagyi Ervin,}\\
.........................
\end{flushright}

%kivonat angol
%-------------------------------------------------------------
\newpage
\thispagestyle{empty}
\begin{center}
    \Large ABSTRACT
\end{center}

english abstract


\begin{flushright}
\textbf{Szilagyi Ervin,}\\
.........................
\end{flushright}

%tartalom jegyzek
%-------------------------------------------------------------
\newpage
\tableofcontents
\newpage
\listofindice
\newpage
\listofindiceen

\selectlanguage{magyar}

%-------------------------------------------------------------
\chapter{Bevezető}
\capterro{Întroducere}
\capteren{Indroduction}

Napjainkban a képfeldolgozás egy eléggé elterjedt kutatási terület. A kutatások célja főleg az információ kinyerésére, gépi látás kivitelézésére irányult. Minderre kiváló megoldást jelentett a deep konvolúciós hálok (ConvNets)\cite{1}\cite{2} sikeres használata növelve ezzel ezek népszerűségét. Fontos megjegyezni, hogy a konvolúciós neuron hálok felfedezése már pár évtizede történt, tehát maga a technológia már régebb ismert volt. Az újrafelfedezésüket és hirtelen népszerűség növekedését annak köszönhetik, hogy az utóbbi években olyan hardveres megoldások jelentek meg amik lehetővé teszik az ilyen típusú hálok létrehozását és működtetését. 
\newline 
\indent
Az Nvidia cég 2007-ben bevezette az Nvidia CUDA platformot\cite{3}. Ez egy komoly, használható fejlesztőkörnyezetett jelentett olyan fejlesztők számára akik nagy méretű adatpárhuzamos algoritmusokat szerettek volna fejleszteni. A CUDA környezet direkt elérhetőséget nyújt a videókártya utasításkészletéhez megengedve ezzel ennek a programozását. Ugyanakkor számos olyan videókártya került piacra ami egyre komolyabb számítási készségekkel bírt. Ezt a lehetőséget értelemszerűen a kutatók ki is használták így számos újabb publikáció és javaslat jelent meg amik neuron hálókat használnak az illető probléma megoldására.
\newline
\indent
A deep konvolúciós hálok népszerűségének növekedésével egyre több olyan fejlesztői környezet jelent meg amiknek célja a mesterséges intelligencia feladatok megoldása. Ilyen könyvtárak például a Caffe\cite{4}, Keras\cite{5}, Theano\cite{6}, Tensorflow\cite{7}, Torch\cite{8} stb. Ezek a környezetekben, habár különböző stílusban de egyazon problémákra hivatottak gyors és egyszerű megoldásokat ajánla ugyanúgy mezei szoftverfejlesztők, mint kutatók számára.
\newline
\indent
Az gépi látás egyik fontos alkalmazási területe a képen levő tárgyak, élőlények emberek felismerése. Ilyen területen a konvolúciós hálok kimagasló teljesítményt nyújtanak, olyannyira, hogy egyes kisérletek szerint ez már nemhogy az emberi látással megegyező, hanem azt felülmúló teljesítméynt nyújtanak\cite{9}. Feltevődik a kérdés, hogyha ennyire szofisztikált a gépi látás, akkor nem-e lehetne használni arra, hogy új képeket alkosson. Amint kiderült erre is alkalmasak. Az általam bemutandó dolgozat is ezt a témát próbálja megcélozni. A gépi látás a tárgyak, élőlények mellett képes felismerni maga a kép művészeti stílusát. Ez elsősorban kihasználható arra, hogy híres művészek alkotásait csoportosítsuk, rendszerezzül\cite{10}, de amint e dolgozatból ki fog derülni, ki lehet használni arra is, hogy egy művészeti stílust egy adott festményről átvigyük egy mindennapi képre, fotóra. 
\newline
\indent
A dolgozatom célja magyar híres festőművészek festészeti stílusát átvenni és ezt alkalmazni midennapi képekre illetve mozgóképekre. Eddigiekben, ahhoz hogy egy mindennapi fényképből művészeti képet varázsoljunk, képszerkesztő szoftverek segítségével lehetett elérni manuálisan. Mindezt egy olyan egyén végezhette, akinek képszerkesztési illetve képmanipulálási szakismere volt adott képszerkesztési szoftverkörnyezetben. Magatól értődik az, hogy ez mozgóképek esetében egy időigényes folyamat. Dolgozatom mindezekre megoldást próbál adni, azáltal, hogy az általam elkészített szoftvert bárki használhatja, nincs szükség különböző képszerkesztői szakértelemre, emellett a folyamat ideje jelentősen csökkenni fog. 

%-------------------------------------------------------------
\chapter{Hasonló rendszerek feltérképezése}
\capterro{Studiu bibliografic}
\capteren{Bibliographic study}
A neuron hálok használata a számítástechnikában nem egy újonnan kialakult terület. Frank Rosenblatt 1958-ban publikált egy olyan mintafelismerő algoritmust\cite{11}, ami egyszerű összeadást és kivonást használva képes volt "tanulni". A rendszer képes volt finomhangolni állapotát a bekövetkező iterációk során. Ezt az alogritmust perceptronnak nevezzük. 1975-ben Paul Werbos bevezette a backpropagation algoritmust\cite{12}, amit a perceptronnal együtt használva megoldotta a perceptron azon problémáját miszerint az csak lineárisan elválasztható osztályokat volt képes kategorizálni. Habár a neuron hálok tanulmányozása eléggé igéretesnek látszott, számítási igényük, komplexitásuk és lassú válaszidejük miatt a kutatók arra következtetésre jutottak, hogy a gyakorlatban még nem lehet alkalmazni őket.
\newline
\indent
Yann LeCun professzor és csapata 1998-ban egy újabb topologiájú hálot vezetett be\cite{13}. A LeNet-5 elnevezésű háló konvoluciós rétegeket is tartalmazott ezért konvolúciós neuron hálonak nevezzük. A publikáció célja kézzel írott számjegyek kategorizálása volt, létrehozva ezáltal a MNIST adatbázist, ami 60000 28x28-as felbontású kézzel írott számjegyet tartalmaz, emellett tartalmaz egy 10000 tagból álló teszthalmazt. A dolgozatban bemutatott LeNet-5 háló 0,7\%-os hiba aránnyal volt képes kategorizálni a számjegyeket, ami messze felülmúlta a többrétegű perceptronos megoldást. 
\newline
\indent
Dave Steinkraus, Patrice Simard és Ian Buck 2005-ben publikált dolgozata\cite{14} letette az alapjait a neuronhálók videokártyán történő programozásának. A videókártyán történő adatpárhuzamos programozás hatalmas performancia növekedést jelentett a processzoron futó neuronhálókkal szemben. Előtérbe kerül a deep learning és a mély konvolúciós hálok használata\cite{1}\cite{2}.
\newline
\indent
Eddigiekben sikerült nagyon pontos felismerő illetve osztályozó rendszereket alkotni. A mély konvolúciós hálok használata azonban nem merül ki ennyiben. 2015-ben publikálásra került egy olyan deep learning-et használó algoritmus, ami képes képes illetve festmények művészeti stílusát átvinni egy másik digitális képre\cite{15}. Mostani dolgozatom is erre a publikációra alapoz, az ebben bemutatott módszereket próbálja alkalmazni illetve továbbfejleszteni. A tanuláshoz egy korábban bevezetett és gépi látáshoz használt, előre betanított neuron hálót használnak fel, a VGG-19-et. Yaroslav Nikulin és Roman Novak tudományos kutatása\cite{16} ezzel szemben eddig ugyanezt a módszert alkalmazta más ismertebb előre betanított hálókra, mint például AlexNet, GoogLeNet vagy VGG-16. Ugyanúgy a VGG-16 háló használata is kíváló eredményeket mutatott míg a GoogLeNet és az AlexNet architektúrájuk miatt komolyabb információvesztéshez vezetnek így a végeredmény nem lesz annyira látványos. Ugyanúgy kísérletek irányultak az eredi eljárás optimalizálására, megjelentek olyan rendszerek amik sajátos, erre a célre betanított neuron hálókat alkalmaznak\cite{17}\cite{18}\cite{19}.
\newline
\indent
2016-ban a Prisma labs inc. kiadta mobilos applikációját Prisma név alatt\cite{20}. Az aplikáció előre megadott ismert festői/grafikai stíusokat alkalmazza a telefon kamerája által készített képekre. Az applikáció az előbbiekben bemutatott kutatásokra alapoz. Ugyanakor fontos megjegyezni, hogy maga a stílus alkalmazását a különböző fotókra nem az okostelefon végzi. A szerkeszteni kívánt képet a telefon felkülde egy szervergépre ami majd válaszként a szerkeztett képet küldi vissza. 
\newline
\indent
Maga stílusátvitel nem csak állóképekre alkalmazható, ezt bizonyította Manuel R., Alexey D., Thomas B. tudományos dolgozata\cite{21}, valamit ezt próbalja megoldani a jelenlegi dolgozatom is. Értelemszerűen egy adott videót több álló képkocka alkot. Viszont ahhoz, hogy látványos művészeti mozgóképet gyátsunk, nem elegendő maga a videót darabokra vágni és minden képkockára alkalmazni a stílust. Erre adott megoldást Manuel R. és társainak kutatása. 


%-------------------------------------------------------------
\chapter{A rendszer}
\capterro{Sistemul}
\capteren{The system}

%-------------------------------------------------------------
\section{Áttekintés}
\secro{Privire de ansamblu asupra}
\secen{Overview}

a rendszer

%----------------------------------------------------------------
\chapter{Összefoglaló}
\capterro{Concluzie}
\capteren{Conclusion}

összefoglaló

% Ábrajegyzék
%---------------------------------
\newpage
 \listoffigures


% thebibliography
%---------------------------------
\newpage
\begin{thebibliography}{1}

\bibitem {1}
Krizhevsky, A., Sutskever, I., and Hinton, G. E. ImageNet classification with deep convolutional neural networks (2012)

\bibitem{2}
Zeiler, M. D. and Fergus, R. Visualizing and understanding convolutional networks (2013)

\bibitem{3}
\url {https://en.wikipedia.org/wiki/CUDA} (2017.04.24)

\bibitem{4}
\url {http://caffe.berkeleyvision.org} (2017.04.24)

\bibitem{5}
\url {https://keras.io/} (2017.04.24)

\bibitem{6}
\url {http://deeplearning.net/software/theano/} (2017.04.24)

\bibitem{7}
\url {https://www.tensorflow.org/} (2017.04.24)

\bibitem{8}
\url {http://torch.ch/} (2017.04.24)

\bibitem{9}
\url {https://computerstories.net/microsoft-computer-outperforms-human-image-recognition-12028} (2017.04.29)

\bibitem{10}
Kevin Alfianto, Mei-Chen Yeh, Kai-Lung Hua - Artist-based Classification via Deep Learning with Multi-scale Weighted Pooling (2016)

\bibitem{11}
Rosenblatt F. - The Perceptron: A Probabilistic Model For Information Storage And Organization In The Brain (1958)

\bibitem{12}
Werbos, P.J. - Beyond Regression: New Tools for Prediction and Analysis in the Behavioral Sciences (1975)
 
\bibitem{13}
LeCun, Yann, Léon Bottou, Yoshua Bengio, Patrick Haffner - Gradient-based learning applied to document recognition (1998)

\bibitem{14}
Dave Steinkraus, Patrice Simard. Ian Buck - Using GPUs for Machine Learning Algorithms (2005)

\bibitem{15}
Leon A. Gatys, Alexander S. Ecker, Matthias Bethge - A neural algorithm of artistic style (2015)

\bibitem{16}
Yaroslav Nikulin, Roman Novak - Exploring the Neural Algorithm of Artistic Style (2016)

\bibitem{17}
Justin Johnson, Alexandre Alahi, Li Fei-Fei - Perceptual Losses for Real-Time Style Transfer and Super-Resolution

\bibitem{18}
Ulyanov, D., Lebedev, V., Vedaldi, A., and Lempitsky - Texture networks: Feed-forward synthesis of textures and stylized images

\bibitem{19}
Ulyanov, D., Lebedev, V., Vedaldi, A., and Lempitsky - Instance Normalization: The Missing Ingredient for Fast Stylization

\bibitem{20}
\url{https://en.wikipedia.org/wiki/Prisma_(app)} (2017.04.29)

\bibitem{21}
Manuel Ruder, Alexey Dosovitskiy, Thomas Brox - Artistic style transfer for videos



\end{thebibliography}

\end{document}
